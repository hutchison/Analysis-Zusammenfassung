\documentclass[%
a4paper,
%empty,	    %keine Seitenzahlen
%a5paper,	% alle weiteren Papierformat einstellbar
10pt,		% Schriftgroe�e (12pt, 11pt (Standard))
leqno,   	% Nummerierung von Gleichungen links
fleqn,		% Ausgabe von Gleichungen linksb�ndig
]
{scrartcl}

%\pagestyle{empty}
%\pagestyle{headings}

%% Deutsche Anpassungen
\usepackage[ngerman]{babel}
\usepackage[T1]{fontenc}
\usepackage[latin1]{inputenc}

%obligatorischer Mathekram:
\usepackage{amssymb,amstext}
\usepackage[sumlimits]{amsmath}
\usepackage{eulervm}
\usepackage[thref,standard]{ntheorem}

%Einstellungen fuer ntheorem

\theoremheaderfont{\normalfont\scshape}
\theorembodyfont{\normalfont}
\theoremseparator{:\newline}
\newtheorem{defin}{Definition}[subsection]

%n�tzliche Extras:
\usepackage{array,
hhline,
longtable,
tabularx,
enumerate,
hyperref,
color,
setspace,
booktabs,
cite,
caption,
lineno,
lastpage,
algorithm,
}

%noetig f�r Querformat
%\usepackage[landscape]{geometry}

%\usepackage[left=1cm,right=1cm,top=1cm,bottom=1cm,includeheadfoot]{geometry}
\usepackage[cm,
%headings,
]{fullpage}

%\usepackage{fancyhdr}
%\pagestyle{fancy}
%\fancyhf{}
%
%\fancyhead[L]{\textbf{ }\\ }
%\fancyhead[C]{\textbf{Serie Nr.}\\ }
%\fancyhead[R]{\textsc{M. Darm�ntzel}\\ LA Gy, 7251955}
%
%% \fancyfoot[L]{}
%% \fancyfoot[C]{}
%\fancyfoot[R]{\thepage / \pageref{LastPage}}
%
%%Linie oben/unten
%\renewcommand{\headrulewidth}{0.0pt}
%\renewcommand{\footrulewidth}{0.0pt}

\parindent 0pt

%% Packages f�r Grafiken & Abbildungen
\usepackage{graphicx}
%\usepackage{subfig}    %%Teilabbildungen in einer Abbildung
%\usepackage{tikz}      %%TeX ist kein Zeichenprogramm
%\usepackage[all]{xy}
\usepackage{pst-all}

\begin{document}

%\pagestyle{empty} 

\section{Analysis 1}

\subsection{Pr�dikatenlogik}

\subsection{Mengenlehre}
kartesischen Produkts

\subsection{Abbildungen}

\begin{defin}[Abbildung]
\ \newline
Eine Abbildung $f$ von einer Menge $\mathbb{X}$ in eine Menge $\mathbb{Y}$ (Notation: $f: \mathbb{X} \mapsto \mathbb{Y}$)
ist eine Zuordnung oder Vorschrift, die jedem Element $x \in \mathbb{X}$ ein eindeutiges Element $f(x) \in \mathbb{Y}$ zuordnet.
\end{defin}

\begin{defin}[Graph]
\ \newline
Der Graph einer Abbildung ist eine Teilmenge $\mathbb{G} \subset \mathbb{X} \times \mathbb{Y}$, die durch 
Graph($f$) := $\{ (x,f(x)) \in \mathbb{X} \times \mathbb{Y}\ |\ x \in \mathbb{X}\}$ beschrieben wird. 
Jedem $x \in \mathbb{X}$ wird genau ein $y \in \mathbb{X}$ zugeordnet (Notation: $y=f(x)$).
\end{defin}

\subsection{K�rperaxiome}

\begin{defin}[K�rper]\label{koerper}
\ \newline
Ein Tripel ($\mathbb{K}$,$+$,$\cdot$) bestehend aus einer Menge $\mathbb{K}$ und zwei bin�ren Verkn�pfungen 

\begin{tabular}{cccl}
$+$ & : & $\mathbb{K} \times \mathbb{K} \rightarrow \mathbb{K}$, & $(x,y) \mapsto x+y$\\
$\cdot$ & : & $\mathbb{K} \times \mathbb{K} \rightarrow \mathbb{K}$, & $(x,y) \mapsto x \cdot y$\\
\end{tabular}

(normalerweise \emph{Addition} und \emph{Multiplikation}) hei�t genau dann K�rper, wenn f�r alle $x$,$y$,$z \in \mathbb{K}$ die 
folgenden Axiome gelten:
\begin{itemize}
\item Axiome der Addition
    \begin{enumerate}
    \item Assoziativit�t: $x + ( y + z ) = ( x + y ) + z$
    \item Kommutativit�t: $x + y = y + x$
    \item Existenz des neutralen Elements: $\exists 0 \in \mathbb{K}: x + 0 = x$
    \item Existenz der inversen Elemente: \emph{Zu jedem $x \in \mathbb{K}$ existiert genau ein Element $-x \in \mathbb{K}$}: $x + (-x) = 0$
    \end{enumerate}
\item Axiome der Multiplikation
    \begin{enumerate}
    \item Assoziativit�t: $x \cdot ( y \cdot z ) = ( x \cdot y ) \cdot z$
    \item Kommutativit�t: $x \cdot y = y \cdot x$
    \item Existenz des neutralen Elements: $\exists 1 \in \mathbb{K},1 \neq 0: x \cdot 1 = x$
    \item Existenz der inversen Elemente: \emph{Zu jedem $x \in \mathbb{K}$ mit $x \neq 0$ existiert genau ein Element $x^{-1} \in \mathbb{K}$
            }: $x \cdot x^{-1} = 1$
    \end{enumerate}
\item Distributivgesetz: $x \cdot ( y + z ) = x \cdot y + x \cdot z$
\end{itemize}
\end{defin}

\begin{Beispiel}
Die Mengen $\mathbb{Q}, \mathbb{R}$ und $\mathbb{C}$ bilden mit "`+"' und "`\ $\cdot$\ "' einen K�rper. Neutrale Elemente sind 0 bzw. $(0,0)$
f�r die Addition und 1 bzw. $(1,0)$ f�r die Multiplikation.
\end{Beispiel}

\begin{defin}[bin�re Relation]\label{brelation}
\ \newline
Eine bin�re Relation $\cal{R}$ auf der Menge $\mathbb{M}$ ist eine Teilmenge von $\mathbb{M} \times \mathbb{M}$, also $\cal{R} \subset \mathbb{M} \times \mathbb{M}$
\end{defin}

\begin{defin}[Relationseigenschaften]\label{eigrelation}
\ \newline
Eine bin�re Relation $\cal{R}$ auf einer Menge $\mathbb{M}$ hei�t
\begin{itemize}
    \item \textbf{reflexiv}, falls $\forall m \in \mathbb{M}: (m,m) \in \cal{R}$
    \item \textbf{symmetrisch}, falls $\forall m,n \in \mathbb{M}: (m,n) \in \cal{R} \Rightarrow $$ (n,m) \in \cal{R}$
    \item \textbf{transitiv}, falls $\forall k,m,n \in \mathbb{M}: (k,m) \in \cal{R}\ \wedge\ $$ (m,n) \in \cal{R} \Rightarrow $$ (k,n) \in \cal{R}$
\end{itemize}
\end{defin}

\begin{defin}[�quivalenzrelation]
\ \newline
$\cal{R}$ ist eine bin�re Relation auf $\mathbb{M}$, welche reflexiv, symmetrisch und transitiv ist. Notation: $x \sim y$.
\end{defin}

\begin{defin}[Ordnungsrelation]
\ \newline
$\cal{R}$ ist eine bin�re Relation auf $\mathbb{M}$, welche reflexiv, antisymmetrisch und transitiv ist. Notation: $x \leq y$.
\end{defin}

\begin{defin}[angeordneter K�rper]
\ \newline
Ein K�rper $(\mathbb{K},+,\cdot)$ wird mit einer Ordnungsrelation $\leq$ zu einem angeordneten K�rper. Elemente aus $\mathbb{K}$ werden damit 
vergleichbar und es gilt $\forall x,y \in \mathbb{K}: x \leq y \vee y \leq x$
\end{defin}

\begin{defin}[Absolutbetrag]
\begin{displaymath}
|x| := \left\{
\begin{array}{cl}
x & x \geq 0\\
-x & x < 0\\
\end{array}
\right.
\end{displaymath}
\end{defin}

\end{document}
